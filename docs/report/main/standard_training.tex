\section{Standard Training}

The net was trained in a standard way, using the 60,000 images of the train and validation sets. The training went on for 30 epochs.

\begin{center}
  \includegraphics[width=0.7\textwidth]{img/train-base.png}
	\label{train-base} 
\end{center}

The tests carried out on the trained model were of two kind. The first test was made using the 10,000 clean samples from the testing set; the second test was made using the adversarial examples, that is the same samples as the first test but with an added perturbation computed as shown in Section 0.4.

\begin{table}[h]
\centering
\begin{tabular}{@{}lll@{}}
\toprule
                               & Clean & Adversarial \\ \midrule
Correctly Predicted            & 98.47 & 3.36        \\
Error                          & 1.53  & 96.64       \\
Confidence                     & 98.59 & 93.82       \\
Confidence Correctly Predicted & 99.04 & 91.67       \\
Confidence Error               & 69.95 & 93.89       \\ \bottomrule
\end{tabular}
\caption{Test results for standard training model.}
\label{standard-test}
\end{table}
\FloatBarrier


As we can see from the results shown in Table~\ref{standard-test}, the classification using the clean samples as test images worked great, with high prediction rate and confidence. The results for the adversarial test, however, behaved as expected, with a large percentage of misclassification with a high confidence.